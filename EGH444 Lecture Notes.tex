%!TEX TS-program = xelatex
%!TEX options = -aux-directory=Debug -shell-escape -file-line-error -interaction=nonstopmode -halt-on-error -synctex=1 "%DOC%"
\documentclass{article}
\input{LaTeX-Submodule/template.tex}

% Additional packages & macros

% Header and footer
\newcommand{\unitName}{Digital Signals and Image Processing}
\newcommand{\unitTime}{Semester 2, 2024}
\newcommand{\unitCoordinator}{Dr Maryam Haghighat}
\newcommand{\documentAuthors}{Tarang Janawalkar}

\fancyhead[L]{\unitName}
\fancyhead[R]{\parbox[b]{9cm}{\raggedleft\leftmark}}
\fancyfoot[C]{\thepage}

% Copyright
\usepackage[
    type={CC},
    modifier={by-nc-sa},
    version={4.0},
    imagewidth={5em},
    hyphenation={raggedright}
]{doclicense}

\date{}

\begin{document}
%
\begin{titlepage}
    \vspace*{\fill}
    \begin{center}
        \LARGE{\textbf{\unitName}} \\[0.1in]
        \normalsize{\unitTime} \\[0.2in]
        \normalsize\textit{\unitCoordinator} \\[0.2in]
        \documentAuthors
    \end{center}
    \vspace*{\fill}
    \doclicenseThis
    \thispagestyle{empty}
\end{titlepage}
\newpage
%
\tableofcontents
\newpage
%
\section{Digital Image Processing}
Digital image processing is the processing of images on a digital
system using algorithms. A digital image is a binary representation of
visual data that is composed of a finite number of elements, each with
a particular location and value.

Image processing methods can be divided into two categories:
\begin{itemize}
    \item Methods where the input and output are images.
    \item Methods where the input is an image and the output is some
          information extracted from the image.
\end{itemize}
\subsection{Elements of Visual Perception}
The human visual system has influenced and contributed to many
advancements in image processing. The human eye has light receptors
called rods and cones. Humans have around \numrange{6}{7} million cones
in each eye that are highly sensitive to colour and fine details. On
the other hand, there are a total of \numrange{75}{150} million rods
across both eyes, that are sensitive to low levels of illumination.
\subsubsection{Image Formation}
Photo camera lenses are fixed in focal length and they focus at various
distances by varying the distance between the lens and imaging plane
(film/chip). The human eye works in the opposite way, where the
distance between the lens and the imaging plane (retina) is fixed, but
the focal length for focus is varied by changing the shape of the lens.
\subsubsection{Brightness Adaptation and Discrimination}
The human eye is capable of discriminating between a wide range of
intensity levels. This range of light intensity levels is on the order
of \(10^{10}\).
\subsection{Image Generation Components}
There are three components to image generation:
\begin{itemize}
    \item \textbf{Object}: The object being imaged.
    \item \textbf{Energy Source}: The source of energy that illuminates the object.
    \item \textbf{Sensor}: The sensor that detects the energy reflected from the object.
\end{itemize}
Depending on the properties of the energy source and object material and
geometry, the emitted energy can be reflected, transmitted, or absorbed.
\subsubsection{Electromagnetic Spectrum}
The main source of energy for imaging is electromagnetic (EM)
radiation. EM radiation consists of propagating sinusoidal waves
characterised by their oscillating frequency \(f\). Using Planck's
equation, the energy of a photon can be calculated as
\begin{equation*}
    E = hf
\end{equation*}
where \(h = \qty{6.62607015E-34}{J.s}\) is Planck's constant and \(f\) is the frequency of the wave.
Given the speed of light \(c = \qty{299792458}{m.s^{-1}}\), we can also calculate the energy using
\begin{equation*}
    E = \frac{hc}{\lambda}.
\end{equation*}
The EM spectrum is divided into regions based on the frequency of the
waves. The visible light spectrum is a small part of the EM spectrum
that is visible to the human eye. The visible light spectrum ranges
from \qty{380}{nm} to \qty{700}{nm}. Earth's atmosphere also blocks
certain parts of the EM spectrum, such as short wavelength UV and X-rays.

\textbf{Perceived colour} (hue) is related to the wavelength of light,
while the \textbf{brightness} is related to the intensity of the
radiation.
\subsubsection{Image Sensors}
Image sensors capture a specific range of the EM spectrum, for example:
\begin{itemize}
    \item RGB sensors capture the visible light spectrum.
    \item Infrared sensors capture the infrared spectrum.
    \item X-ray sensors capture the X-ray spectrum.
    \item Ultraviolet sensors capture the ultraviolet spectrum.
\end{itemize}
\subsubsection{Human Perception}
Human perception is context-dependent. Perceived intensity around
regions of discontinuous intensity appear to undershoot and overshoot
around the boundary (see Mach band effect). The eye can also fill in
non-existing information and wrongly perceive geometrical properties of
objects. To produce a powerful vision system, we need both a powerful
image sensor and image processor to extract useful information from an
image.
\subsection{Image Sensing and Acquisition}
Image sensing is the process of transforming illuminated energy into a
digital image. The process involves the following steps:
\begin{enumerate}
    \item Convert the illuminated energy into an electrical signal.
    \item Digitize the electrical signal to obtain a digital image.
\end{enumerate}
\subsubsection{Image Sensing Modalities}
Image sensing is done using three principal modalities:
\begin{itemize}
    \item \textbf{Single Sensing Element}: A single sensor that
          captures the energy. For example, a photodiode. To
          generate 2D images, the sensor must be appropriately
          displaced in the \(x\) and \(y\) directions.
    \item \textbf{Line Sensor}: A sensor that captures energy
          along a line. To generate 2D images, the sensor must
          be displaced in the direction perpendicular to the
          line.
    \item \textbf{Array Sensor}: A sensor that captures energy
          in a 2D array. The sensor is divided into rows and
          columns, with each element capturing energy at a
          specific location. A typical arrangement is the
          \textbf{CCD} (Charge Coupled Device) sensor.
\end{itemize}
\subsubsection{Image Formation}
Let us denote the intensity of a monochrome image by the 2-dimensional
function
\begin{equation*}
    \ell = f\left( x,\: y \right)
\end{equation*}
where \(x\) and \(y\) represent the spatial coordinates captured by the
sensor, and \(f\) is a scalar function of the intensity of the energy
radiated by a physical source. As such, this function is non-negative
and finite:
\begin{equation*}
    0 \leqslant \ell \leqslant \infty.
\end{equation*}
\(f\) is characterised by two components:
\begin{itemize}
    \item \textbf{Illumination}: The amount of source illumination incident on the
          scene \(i\left( x,\: y \right)\). Here \(0 \leqslant i\left( x,\: y \right) < \infty\).
    \item \textbf{Reflectance}: The amount of illumination reflected by
          the objects on the scene \(r\left( x,\: y \right)\). Here
          \(0 \leqslant r\left( x,\: y \right) \leqslant 1\).
\end{itemize}
Therefore, we can describe the image formation process as
\begin{equation*}
    f\left( x,\: y \right) = i\left( x,\: y \right) r\left( x,\: y \right),
\end{equation*}
where \(r = 0\) implies total absorption, while \(r = 1\) implies total reflectance.
For monochrome images, we can define the minimum and maximum intensity
values as \(L_{\text{min}}\) and \(L_{\text{max}}\), respectively,
where
\begin{equation*}
    L_{\text{min}} \leqslant \ell \leqslant L_{\text{max}}, \quad L_{\text{min}} = i_{\text{min}} r_{\text{min}}, \quad L_{\text{max}} = i_{\text{max}} r_{\text{max}}.
\end{equation*}
The range of intensity values \(\interval{L_{\text{min}}}{L_{\text{max}}}\)
is called the \textbf{intensity/gray scale}. Commonly, this interval is
transformed to the interval \(\interval{0}{L-1}\), where \(L\) is the
number of intensity levels.
\subsection{Image Sampling and Quantisation}
\textbf{Image sampling} is the process of sampling discrete points in a
continuous image. Regardless of the sensor arrangement, the image is
sampled at a fixed rate in the \(x\) and \(y\) directions and the
resulting points are called \textbf{pixels}. These pixels are stored in
an array that is \(M \times N\) in size, where \(M\) is the number of
rows and \(N\) is the number of columns, arranged as shown below:
\begin{equation*}
    \symbf{X} =
    \begin{bmatrix}
        x_{00}    & x_{01}    & \cdots & x_{0,N-1}   \\
        x_{10}    & x_{11}    & \cdots & x_{1,N-1}   \\
        \vdots    & \vdots    & \ddots & \vdots      \\
        x_{M-1,0} & x_{M-1,1} & \cdots & x_{M-1,N-1}
    \end{bmatrix}
\end{equation*}
\textbf{Image quantisation} is the process of converting the continuous
intensity values of an image to discrete values. The number of
intensity levels \(L\) is determined by the number of bits used to
represent each pixel. The number of intensity levels is given by
\begin{equation*}
    L = 2^k,
\end{equation*}
where \(k\) is the bit depth. Thus the quantised intensity values will
range from \(\interval{0}{L-1}\). The \textbf{quality} of an image is
determine by the number of discrete intensity levels used in both
sampling and quantisation.
\subsubsection{Dynamic Range and Contrast}
The \textbf{dynamic range} of an image is the ratio of the maximum
intensity value to the minimum intensity value:
\begin{equation*}
    \text{Dynamic Range} = \frac{L_{\text{max}}}{L_{\text{min}}} = \frac{i_{\text{max}} r_{\text{max}}}{i_{\text{min}} r_{\text{min}}}.
\end{equation*}
The upper limit is determined by the sensor's saturation level, while
the lower limit is determined by the sensor's noise level. The
\textbf{contrast} of an image is the difference in intensity between the
brightest and darkest regions of the image:
\begin{equation*}
    \text{Contrast} = L_{\text{max}} - L_{\text{min}} = i_{\text{max}} r_{\text{max}} - i_{\text{min}} r_{\text{min}}.
\end{equation*}
\begin{itemize}
    \item A \textbf{high} dynamic range implies a large difference
          between the brightest and darkest regions of the image, and
          therefore high contrast.
    \item A \textbf{low} dynamic range implies a small difference
          between the brightest and darkest regions of the image, and
          therefore low contrast.
\end{itemize}
\subsubsection{Spatial and Intensity Resolution}
The \textbf{spatial resolution} of an image is a measure of the
smallest discernible detail in an image, measured as the number of
pixels per unit area (dots per inch). In some images, sampling an image
at a low rate can result in \textbf{aliasing}, where high-frequency
components are incorrectly represented as low-frequency components (see
the Moiré pattern).

The \textbf{intensity resolution} of an image is the smallest
discernible change in intensity level, which is related to the number
of intensity levels used to represent the image, for example, 8-bit and
10-bit images. Choosing a low number of intensity levels can result in
quantisation noise, where intensity levels are incorrectly represented.
\subsubsection{Image Interpolation}
Image interpolation is the process of estimating the intensity values
of pixels between the sampled points. Interpolation is used to increase
and decrease the resolution of an image for resampling and resizing.
Common interpolation methods include:
\begin{itemize}
    \item \textbf{Nearest Neighbour}: The intensity value of the nearest
          pixel is used to estimate the intensity value of the pixel.
    \item \textbf{Bilinear}: The intensity value of the nearest four
          pixels is used to estimate the intensity value of the pixel.
    \item \textbf{Bicubic}: The intensity value of the nearest sixteen
          pixels is used to estimate the intensity value of the pixel.
\end{itemize}
\subsection{Relationships Between Pixels}
The following sections will define some common sets that are used to
describe relationships between pixels in an image.
\subsubsection{Neighbours of a Pixel}
The \textbf{neighbours} of a pixel \(\symbf{p} = \left( x,\: y
\right)\) are the pixels that are adjacent to \(\symbf{p}\).
\begin{itemize}
    \item The \textbf{4-neighbours} of \(\symbf{p}\) are defined as the
          pixels that are adjacent to \(\symbf{p}\) in the
          \textbf{cardinal} directions:
          \begin{equation*}
              \symbf{N}_4\left( \symbf{p} \right) = \left\{ \left( x,\: y-1 \right), \left( x-1,\: y \right), \left( x+1,\: y \right), \left( x,\: y+1 \right) \right\},
          \end{equation*}
    \item The \textbf{diagonal-neighbours} of \(\symbf{p}\) are defined
          as the pixels that are adjacent to \(\symbf{p}\) in the
          \textbf{diagonal} directions:
          \begin{equation*}
              \symbf{N}_D\left( \symbf{p} \right) = \left\{ \left( x-1,\: y-1 \right), \left( x+1,\: y-1 \right), \left( x-1,\: y+1 \right), \left( x+1,\: y+1 \right) \right\},
          \end{equation*}
    \item The \textbf{8-neighbours} of \(\symbf{p}\) are defined as the
          pixels that are adjacent to \(\symbf{p}\) in both the
          \textbf{cardinal} and \textbf{diagonal} directions:
          \begin{equation*}
              \symbf{N}_8\left( \symbf{p} \right) = \symbf{N}_4\left( \symbf{p} \right) \cup \symbf{N}_D\left( \symbf{p} \right).
          \end{equation*}
\end{itemize}
\subsubsection{Adjacency and Connectivity}
Two pixels \(\symbf{p}\) and \(\symbf{q}\) are \textbf{adjacent} if
they are neighbours and their intensity values are similar or belong to
the same set of values \(V\) based on a threshold. This can occur in
one of three ways:
\begin{itemize}
    \item \textbf{4-Adjacency} when \(\symbf{q} \in \symbf{N}_4\left( \symbf{p} \right)\).
    \item \textbf{8-Adjacency} when \(\symbf{q} \in \symbf{N}_8\left( \symbf{p} \right)\).
    \item \textbf{M-Adjacency} when \(\symbf{q} \in \symbf{N}_4\left( \symbf{p} \right)\) \textbf{or} \(\symbf{q} \in \symbf{N}_D\left( \symbf{p} \right)\) \textbf{and} \(\symbf{N}_4\left( \symbf{p} \right) \cap \symbf{N}_4\left( \symbf{q} \right) = \emptyset\).
          This statement avoids double-counting an adjacency when
          \(\symbf{q}\) is a diagonal neighbour of \(\symbf{p}\) while
          another cardinal neighbour exists between \(\symbf{p}\) and
          \(\symbf{q}\). Consider the binary example with \(\symbf{p},\symbf{q},\symbf{r} \in V = \left\{ 1 \right\}\):
          \begin{equation*}
              \begin{bmatrix}
                  0 & \symbf{r} & \symbf{q} \\
                  0 & \symbf{p} & 0         \\
                  0 & 0         & 0
              \end{bmatrix}
              .
          \end{equation*}
          Here we do not consider \(\symbf{p}\) and \(\symbf{q}\) to be
          M-adjacent, as \(\symbf{p}\) and \(\symbf{q}\) are already
          adjacent through \(\symbf{r}\).
\end{itemize}
A \textbf{path} (or curve) between two pixels \(\symbf{p}\) and
\(\symbf{q}\) is a sequence of \(n+1\) pixels \(\left( \symbf{p}_0, \symbf{p}_1,
\ldots, \symbf{p}_n \right)\) such that \(\symbf{p}_0 = \symbf{p}\) and
\(\symbf{p}_n = \symbf{q}\), where \(\symbf{p}_i\) is adjacent to
\(\symbf{p}_{i+1}\) for \(i = 0, 1, 2, \ldots, n-1\). This path is said
to have length \(n\), and is \textit{closed} if \(\symbf{p}_0 = \symbf{p}_n\).
\subsubsection{Connectivity}
Consider a subset of pixels in an image \(S\) with \(\symbf{p},
\symbf{q} \in S\).
\begin{itemize}
    \item \(\symbf{p}\) and \(\symbf{q}\) are \textbf{connected} if
          there exists a path between \(\symbf{p}\) and \(\symbf{q}\) such
          that all pixels in the path are in \(S\).
    \item The set of pixels connected to \(\symbf{p}\) in \(S\) form a
          \textbf{connected component}.
    \item If \(S\) only consists of \textit{one} connected component,
          then \(S\) is said to be a \textbf{connected set} and is
          called a \textbf{region} \(R\).
\end{itemize}
Two regions \(R_1\) and \(R_2\) are adjacent if their union forms a
connected set, i.e., another region. Regions that are not adjacent are
said to be \textbf{disjoint}.
\subsection{Distance Metrics}
The \textbf{distance} between two pixels \(\symbf{p}\) and
\(\symbf{q}\) can be measured using a variety of metrics. The most
common metrics are:
\begin{itemize}
    \item \textbf{Euclidean distance}:
          \begin{equation*}
              d\left( \symbf{p},\: \symbf{q} \right) = \sqrt{\left( x_p - x_q \right)^2 + \left( y_p - y_q \right)^2}.
          \end{equation*}
    \item \textbf{Manhattan (D4) distance} (or city-block distance):
          \begin{equation*}
              d_4\left( \symbf{p},\: \symbf{q} \right) = \abs{x_p - x_q} + \abs{y_p - y_q}.
          \end{equation*}
    \item \textbf{Chessboard (D8) distance} (or maximum distance):
          \begin{equation*}
              d_8\left( \symbf{p},\: \symbf{q} \right) = \max\left( \abs{x_p - x_q},\: \abs{y_p - y_q} \right).
          \end{equation*}
\end{itemize}
In general, any metric \(d\) that satisfies the following properties is a
\textbf{distance metric}:
\begin{itemize}
    \item The distance from a point to itself is zero:
          \begin{equation*}
              d\left( \symbf{p},\: \symbf{p} \right) = 0.
          \end{equation*}
    \item \textbf{Positivity}: The distance between two distinct points is always positive:
          \begin{equation*}
              d\left( \symbf{p},\: \symbf{q} \right) > 0 : \symbf{p} \neq \symbf{q}.
          \end{equation*}
    \item \textbf{Symmetry}: The distance between two points is always the same regardless of ordering:
          \begin{equation*}
              d\left( \symbf{p},\: \symbf{q} \right) = d\left( \symbf{q},\: \symbf{p} \right).
          \end{equation*}
    \item The Triangle Inequality is satisfied:
          \begin{equation*}
              d\left( \symbf{p},\: \symbf{q} \right) \leq d\left( \symbf{p},\: \symbf{r} \right) + d\left( \symbf{r},\: \symbf{q} \right).
          \end{equation*}
\end{itemize}
for all \(\symbf{p},\: \symbf{q},\: \symbf{r}\) in this metric space.
\subsection{Mathematical Operations}
\subsubsection{Element-wise Operations}
Given two images \(\symbf{X}\) and \(\symbf{Y}\) of equal dimensions,
an element-wise operation is an operation that is applied to each pixel
in the image. Suppose we wish to apply the binary operation \(\otimes\)
on \(\symbf{X}\) and \(\symbf{Y}\). The resulting image \(\symbf{Z}\)
is given by
\begin{equation*}
    \symbf{Z} = \symbf{X} \otimes \symbf{Y} =
    \begin{bmatrix}
        x_{00} \otimes y_{00}       & x_{01} \otimes y_{01}       & \cdots & x_{0,N-1} \otimes y_{0,N-1}     \\
        x_{10} \otimes y_{10}       & x_{11} \otimes y_{11}       & \cdots & x_{1,N-1} \otimes y_{1,N-1}     \\
        \vdots                      & \vdots                      & \ddots & \vdots                          \\
        x_{M-1,0} \otimes y_{M-1,0} & x_{M-1,1} \otimes y_{M-1,1} & \cdots & x_{M-1,N-1} \otimes y_{M-1,N-1}
    \end{bmatrix}
\end{equation*}
Here the operator \(\otimes\) can represent any binary operation, such
as addition, subtraction, multiplication, or division.
\subsubsection{Linear Operations}
The operator \(H\) is a linear operation if it satisfies the following
property:
\begin{equation*}
    H\left[ a\symbf{X} + b\symbf{Y} \right] = aH\left[ \symbf{X} \right] + bH\left[ \symbf{Y} \right],
\end{equation*}
for images \(\symbf{X}\) and \(\symbf{Y}\), where \(a\) and \(b\) are constants.
\subsubsection{Spatial Operations}
The operator \(T\) is a spatial operation if the output pixel value is
determined by the values of pixels in the neighbourhood of any input
pixel. \(T\) can be categorised as one of three types of spatial
operations:
\begin{itemize}
    \item \textbf{Pointwise}: The output pixel value is
          determined by the value of the input pixel only.
    \item \textbf{Neighbourhood}: The output pixel value is
          determined by the value of the input pixel and any
          neighbouring pixels.
    \item \textbf{Global}: The output pixel value is
          determined by the value of pixels in the entire image.
    \item \textbf{Geometric}: The output pixel value is
          determined by the spatial location of the pixel.
\end{itemize}
\section{Intensity Transformations \& Spatial Filters}
\subsection{Intensity Transformations}
An intensity transform aims to modify the contrast of an image by
changing the range of intensity values in that image. The following
sections will define some common intensity transformations. In the
following sections let \(r\) be the input intensity and \(s\) be the
output intensity of an image.
\subsubsection{Identity Transformation}
The identity transformation is the simplest intensity transformation
which does not alter an image. It is defined as
\begin{equation*}
    s = T\left( r \right) = r.
\end{equation*}
\subsubsection{Negative Transformation}
The image negation transformation is used to invert the intensity
values of an image. It is defined as
\begin{equation*}
    s = T\left( r \right) = \left( L - 1 \right) - r,
\end{equation*}
where \(L\) is the number of intensity levels in the image.
\subsubsection{Logarithmic Transformation}
The logarithmic transformation is used to enhance darker regions of an
image by compressing brighter regions. It is defined as
\begin{equation*}
    s = T\left( r \right) = c \log{\left( 1 + r \right)},
\end{equation*}
where \(c\) is a constant that scales the intensity values of the image.
\subsubsection{Power-Law (Gamma) Transformation}
The power-law transformation is used to correct the gamma of an image,
either by enhancing or reducing dark or bright regions. It can be
thought of as a generalisation of the logarithmic transformation. It is
defined as
\begin{equation*}
    s = T\left( r \right) = c r^\gamma,
\end{equation*}
where \(c\) is a constant that scales the intensity values of the image
and \(\gamma\) is the gamma value.
\begin{itemize}
    \item When \(\gamma < 1\), the transformation enhances the darker
          regions of the image, while compressing the brighter regions.
    \item When \(\gamma > 1\), the transformation enhances the brighter
          regions of the image, while compressing the darker regions.
\end{itemize}
\subsubsection{Piecewise-Linear Transformation}
Piecewise-linear transformations are used to enhance the contrast of
specific regions of an image. Some common piecewise-linear
transformations include:
\begin{itemize}
    \item \textbf{Contrast Stretching}: Enhances the contrast of an
          image by stretching the intensity values to the full range of
          intensity levels.
    \item \textbf{Intensity Level Slicing}: Enhances the contrast of
          specific regions of an image by setting the intensity values
          of other regions to zero or by leaving them unchanged.
    \item \textbf{Bit-Plane Slicing}: Highlights the contribution made
          to image appearance by specific bits in the image.
\end{itemize}
\subsection{Histogram Processing}
Histograms are used to visualise the distribution of intensity values
in an image. Given an image \(\symbf{X} \in \interval{0}{L-1}^{M \times
N}\), the histogram \(h_{\symbf{X}}\left( k \right)\) is defined as
\begin{equation*}
    h_{\symbf{X}}\left( k \right) = n_k,
\end{equation*}
where \(n_k\) is the number of pixels in the image with intensity value
\(k\). If we normalise these values, we find the probability of
obtaining a pixel with intensity value \(k\):
\begin{equation*}
    p_{\symbf{X}}\left( k \right) = \frac{n_k}{MN}.
\end{equation*}
It follows that
\begin{equation*}
    \sum_{k=0}^{L-1} p_{\symbf{X}}\left( k \right) = 1.
\end{equation*}
\subsubsection{Histogram Equalisation}
Histogram equalisation is a method used to spread the most frequent
intensity values in an image to the full range of intensity levels,
thereby achieving a more uniform distribution of intensity values. To
do this, we will use the following transformation that maps the
cumulative distribution function of an input image \(\symbf{X}\) to the
cumulative distribution function of a uniform distribution:
\begin{equation*}
    s = T\left( r \right) = \left( L - 1 \right) \sum_{j=0}^{r} p_{\symbf{X}}\left( j \right) = \frac{L - 1}{MN} \sum_{j=0}^{r} n_j.
\end{equation*}
\subsubsection{Histogram Matching}
In some cases, we wish to match the histogram of \(\symbf{X}\) to the
histogram of another image \(\symbf{Y}\). To do so, consider the
histogram equalisation of \(\symbf{Y}\) with intensity values \(z\):
\begin{equation*}
    s = G\left( z \right) = \left( L - 1 \right) \sum_{j=0}^{z} p_{\symbf{Y}}\left( j \right) = \frac{L - 1}{MN} \sum_{j=0}^{z} n_j.
\end{equation*}
Thus we have the mapping:
\begin{equation*}
    T: r \mapsto s \quad \text{and} \quad G: z \mapsto s.
\end{equation*}
As both \(T\) and \(G\) map to the same equalised space, we can define
the transformation \(z = H\left( r \right)\) that maps the histogram of
\(\symbf{X}\) to the histogram of \(\symbf{Y}\) as
\begin{equation*}
    z = H\left( r \right) = G^{-1}\left( T\left( r \right) \right) = G^{-1}\left( s \right).
\end{equation*}
\subsection{Spatial Filtering}
Spatial filtering is the process of creating a new image by applying a
mask (or kernel, template or window) to each pixel in an image. This
new pixel value is determined by the intensity values of the pixels in
the neighbourhood of the original pixel. The mask is defined as a \(m
\times n\) matrix \(\symbf{W}\) with elements \(w_{ij}\) that represent
the weights of the pixels in the neighbourhood of the pixel being
processed. For convenience, \(m\) and \(n\) are typically odd integers.

The output pixel value is given by the weighted sum of the intensity
values of the pixels in the neighbourhood of the pixel. This can be
done using one of two operations:
\begin{itemize}
    \item \textbf{Correlation}: The mask is shifted across the image
          and the weighted sum is calculated at each position.
    \item \textbf{Convolution}: The mask is first flipped horizontally
          and vertically before it is shifted across the image.
\end{itemize}
\subsubsection{Correlation}
The \(\left( i,\: j \right)\)th element of the correlation of an image
\(\symbf{X}\) with a mask \(\symbf{W}\) is defined as
\begin{equation*}
    y_{ij} = \symbf{W} \star \symbf{X} = \sum_{s = -a}^a \sum_{t = -b}^b w_{st} x_{i+s, j+t}
\end{equation*}
for \(a = \frac{m-1}{2}\) and \(b = \frac{n-1}{2}\).
\subsubsection{Convolution}
The \(\left( i,\: j \right)\)th element of the convolution of an image
\(\symbf{X}\) with a mask \(\symbf{W}\) is defined as
\begin{equation*}
    y_{ij} = \symbf{W} \ast \symbf{X} = \sum_{s = -a}^a \sum_{t = -b}^b w_{st} x_{i-s, j-t}
\end{equation*}
for \(a = \frac{m-1}{2}\) and \(b = \frac{n-1}{2}\).
\subsubsection{Padding}
For masks larger than \(1 \times 1\), the indices \(w_{st}\) will
exceed the bounds of the image \(\symbf{X}\). To prevent this, the
image is often padded with an additional border of pixels. Common
padding methods include:
\begin{itemize}
    \item \textbf{Zero Padding}: The border is padded with zeros.
    \item \textbf{Boundary Replication Padding}: The border is padded with the
          intensity values of the nearest pixel.
    \item \textbf{Reflection Padding}: The border is padded with the
          intensity values of reflected pixels (one pixel away from the
          border).
\end{itemize}
\subsubsection{Averaging Filters}
Averaging filters are used to reduce noise in an image by averaging the
intensity values of the pixels in the neighbourhood of the pixel being
processed. An averaging filter considers a continuous function of two
variables, such as the multivariable Gaussian function:
\begin{equation*}
    w_{st} = \frac{1}{2\pi\sigma^2} \exp{\left( -\frac{s^2 + t^2}{2\sigma^2} \right)}.
\end{equation*}
\subsubsection{Smoothing Linear Filters}
Smoothing filters are used to reduce noise in an image by averaging the
intensity values of the pixels in the neighbourhood of the pixel being
processed. A general implementation for filtering an \(M \times N\)
image with a \textbf{weighted averaging filter} of size \(m \times n\)
is defined as:
\begin{equation*}
    y_{ij} = \frac{\sum_{s = -a}^a \sum_{t = -b}^b w_{st} x_{i+s,j+t}}{\sum_{s = -a}^a \sum_{t = -b}^b w_{st}},
\end{equation*}
\subsubsection{Order-Statistic Non-Linear Filters}
Order-statistic filters are used to reduce noise in an image by
replacing the intensity value of a pixel with the median, maximum, or
minimum intensity value of the pixels in the neighbourhood of the pixel
being processed. Median filters have good noise-reduction capabilities
with less smoothing and are used to remove impulse or salt-and-pepper
noise.
\subsubsection{Sharpening Filters}
Sharpening filters are used to enhance edges and discontinuities in an
image. One technique is to consider the Laplacian of the image
\(f\left( x,\: y \right)\):
\begin{equation*}
    \nabla^2 f = \pdv[order=2]{f}{x} + \pdv[order=2]{f}{y}.
\end{equation*}
For discrete images, we will use the first-order forward difference
approximation:
\begin{align*}
    \pdv{f}{x} & = f\left( x + 1,\: y \right) - f\left( x,\: y \right)  \\
    \pdv{f}{y} & = f\left( x,\: y + 1 \right) - f\left( x,\: y \right),
\end{align*}
and the second-order central difference approximation:
\begin{align*}
    \pdv[order=2]{f}{x} & = f\left( x + 1,\: y \right) - 2f\left( x,\: y \right) + f\left( x - 1,\: y \right)  \\
    \pdv[order=2]{f}{y} & = f\left( x,\: y + 1 \right) - 2f\left( x,\: y \right) + f\left( x,\: y - 1 \right).
\end{align*}
The Laplacian allows us to identify transitions in intensity values
across an image by creating a filter with one of the following masks:
\begin{equation*}
    \symbf{W} =
    \begin{bmatrix}
        0 & 1  & 0 \\
        1 & -4 & 1 \\
        0 & 1  & 0
    \end{bmatrix}
    \quad \text{or} \quad \symbf{W} =
    \begin{bmatrix}
        0  & -1 & 0  \\
        -1 & 4  & -1 \\
        0  & -1 & 0
    \end{bmatrix}
    ,
\end{equation*}
where the first mask will sharpen the image, while the second mask will
sharpen the negative of the image. In general, we can sharpen edges by
adding the Laplacian to the original image:
\begin{equation*}
    g\left( x,\: y \right) = f\left( x,\: y \right) + c \nabla^2 f\left( x,\: y \right),
\end{equation*}
where \(c = -1\) will sharpen the image, while \(c = 1\) sharpen the
negative of the image.
\subsubsection{Unsharp Masking and High-Boost Filtering}
Unsharp masking is a sharpening technique that enhances edges and
discontinuities in an image by subtracting a blurred version of the
image from the original image. The process takes the following steps:
\begin{enumerate}
    \item Blur the original image \(f\left( x,\: y \right)\).
    \item Subtract the blurred image from the original image to obtain the mask
    \begin{equation*}
        g_{\text{mask}}\left( x,\: y \right) = f\left( x,\: y \right) - \bar{f}\left( x,\: y \right),
    \end{equation*}
    where \(\bar{f}\left( x,\: y \right)\) is the blurred image.
    \item Add the mask to the original image to obtain a sharpened image:
    \begin{equation*}
        g\left( x,\: y \right) = f\left( x,\: y \right) + k g_{\text{mask}}\left( x,\: y \right), \quad k > 0.
    \end{equation*}
    where
    \begin{itemize}
        \item \(k = 1\) corresponds to unsharp masking.
        \item \(k > 1\) corresponds to high-boost filtering.
    \end{itemize}
\end{enumerate}
\section{Filtering in the Frequency Domain}
\subsection{2D Fourier Transform}
The 2D Fourier transform of an image \(f\left( x,\: y \right)\)\footnote{\(f\) must satisfy the Dirichlet conditions for the 2D Fourier transform to exist.}
is defined as
\begin{equation*}
    F\left( u,\: v \right) = \mathscr{F}\left\{ f\left( x,\: y \right) \right\} = \int_{-\infty}^{\infty} \int_{-\infty}^{\infty} f\left( x,\: y \right) e^{-j2\pi\left( ux + vy \right)} \odif{x} \odif{y},
\end{equation*}
with the inverse 2D Fourier transform defined as:
\begin{equation*}
    f\left( x,\: y \right) = \mathscr{F}^{-1}\left\{ F\left( u,\: v \right) \right\} = \int_{-\infty}^{\infty} \int_{-\infty}^{\infty} F\left( u,\: v \right) e^{j2\pi\left( ux + vy \right)} \odif{u} \odif{v}.
\end{equation*}
When \(f\) is a discrete \(M \times N\) image, the 2D discrete Fourier
transform is defined as
\begin{equation*}
    F\left( u,\: v \right) = \sum_{x = 0}^{M-1} \sum_{y = 0}^{N-1} f\left( x,\: y \right) e^{-j2\pi\left( ux/M + vy/N \right)},
\end{equation*}
with the inverse 2D discrete Fourier transform defined as
\begin{equation*}
    f\left( x,\: y \right) = \sum_{u = 0}^{M-1} \sum_{v = 0}^{N-1} F\left( u,\: v \right) e^{j2\pi\left( ux/M + vy/N \right)}.
\end{equation*}
Many properties of the Fourier transform in the 1D case extend to the 2D
case, such as linearity, symmetry, and shift properties. Notably, the
sampling theorem in 2D states that an image can be perfectly
reconstructed from its samples if the sampling rate \(f_s\) is greater
than twice the highest frequency component in the image:
\begin{equation*}
    f_s > 2f_{\text{max}}.
\end{equation*}
In 2D, this is equivalent to:
\begin{equation*}
    \frac{1}{\adif{x}} > 2 u_{\text{max}} \quad \text{and} \quad \frac{1}{\adif{y}} > 2 v_{\text{max}},
\end{equation*}
where \(u_{\text{max}}\) and \(v_{\text{max}}\) are the maximum
frequencies in the \(x\) and \(y\) directions, respectively. The
frequency of spatial and frequency samples can also be related by the
following equations:
\begin{equation*}
    \adif{u} = \frac{1}{M\adif{x}} \quad \text{and} \quad \adif{v} = \frac{1}{N\adif{y}}.
\end{equation*}
\subsection{Filtering in the Frequency Domain}
Filtering can also be performed in the frequency domain if the image
contains noise that is more easily removed when visualising the
magnitude and phase spectra of the image. In general, a filter in the
frequency domain is defined as
\begin{equation*}
    g\left( x,\: y \right) = \mathscr{F}^{-1}\left\{ H\left( u,\: v \right) F\left( u,\: v \right) \right\},
\end{equation*}
where \(H\left( u,\: v \right)\) is a filter designed to act on the
frequency spectrum of the image. Due to the lack of padding, the
horizontal or vertical edges of the resulting image may contain black
pixels, leading to inconsistent filtering. To avoid this, we can take
the following steps:
\begin{enumerate}
    \item For an image \(f\left( x,\: y \right)\) of size \(M \times N\), define padding parameters
          \(P = 2M\) and \(Q = 2N\).
    \item Form a padded image \(f_p\left( x,\: y \right)\) of size \(P \times Q\)
          by appending the necessary number of zeros to \(f\).\
    \item Multiply the padded image by \(\left( -1 \right)^{x + y}\) to centre the
          transform at the origin.
    \item Compute the 2D Fourier transform of the transformed padded image.
    \item Multiply the result by the symmetric filter \(H\left( u,\: v \right)\) of size \(P \times Q\) to form \(G\left( u,\: v \right)\).
    \item Obtain the processed image \(g_p\left( x,\: y \right) = \Re\left\{ \mathscr{F}^{-1}\left\{ G\left( u,\: v \right) \right\} \right\} \left( -1 \right)^{x + y}\).
    \item Obtain the final processed image \(g\left( x,\: y \right)\) by
          cropping the processed image to the original size \(M \times N\),
          which is in the top-left corner of the processed image.
\end{enumerate}
\subsection{Image Smoothing}
\subsubsection{Ideal Low-Pass Filters}
An ideal low-pass filter passes all frequencies within a circle of
radius \(D_0\) from the origin and suppresses all frequencies outside
of this circle, without attenuation:
\begin{equation*}
    H\left( u,\: v \right) =
    \begin{cases}
        1 & \text{if } D\left( u,\: v \right) \leq D_0, \\
        0 & \text{if } D\left( u,\: v \right) > D_0,
    \end{cases}
\end{equation*}
Here \(D\left( u,\: v \right)\) is the distance from the origin to the
point \(\left( u,\: v \right)\) in the frequency domain:
\begin{equation*}
    D\left( u,\: v \right) = \sqrt{\left( u - P/2 \right)^2 + \left( v - Q/2 \right)^2}.
\end{equation*}
\subsubsection{Butterworth Low-Pass Filters}
A Butterworth low-pass filter is defined as
\begin{equation*}
    H\left( u,\: v \right) = \frac{1}{1 + \left( D\left( u,\: v \right)/D_0 \right)^{2n}},
\end{equation*}
where \(n\) is the order of the filter. A higher-order filter has a
sharper transition between the passband and the stopband, but also
results in more rippling in the spatial domain.
\subsubsection{Gaussian Low-Pass Filters}
A Gaussian low-pass filter is defined as
\begin{equation*}
    H\left( u,\: v \right) = \exp{\left( -\frac{D^2\left( u,\: v \right)}{2D_0^2} \right)},
\end{equation*}
where \(D_0\) is the standard deviation of the Gaussian filter.
\subsection{Image Sharpening}
\subsubsection{Ideal High-Pass Filters}
An ideal high-pass filter passes all frequencies outside a circle of
radius \(D_0\) from the origin and suppresses all frequencies inside of
this circle, without attenuation:
\begin{equation*}
    H\left( u,\: v \right) =
    \begin{cases}
        0 & \text{if } D\left( u,\: v \right) \leq D_0, \\
        1 & \text{if } D\left( u,\: v \right) > D_0.
    \end{cases}
\end{equation*}
High-pass equivalent filters can be generated for all of the above
low-pass filters by taking the complement of the low-pass filter:
\begin{equation*}
    H_{\text{HP}}\left( u,\: v \right) = 1 - H_{\text{LP}}\left( u,\: v \right).
\end{equation*}
\subsubsection{Butterworth High-Pass Filters}
A Butterworth high-pass filter is defined as
\begin{equation*}
    H\left( u,\: v \right) = \frac{1}{1 + \left( D_0/D\left( u,\: v \right) \right)^{2n}},
\end{equation*}
where \(n\) is the order of the filter.
\subsubsection{Gaussian High-Pass Filters}
A Gaussian high-pass filter is defined as
\begin{equation*}
    H\left( u,\: v \right) = 1 - \exp{\left( -\frac{D^2\left( u,\: v \right)}{2D_0^2} \right)},
\end{equation*}
where \(D_0\) is the standard deviation of the Gaussian filter.
\subsubsection{Laplacian Filters}
We can consider an alternative formulation for the Laplacian using a
filter the frequency domain. In the frequency domain, the Laplacian
becomes
\begin{equation*}
    \nabla^2 f = \pdv[order=2]{f}{x} + \pdv[order=2]{f}{y} = -4\pi^2 u^2 F\left( u,\: v \right) - 4\pi^2 v^2 F\left( u,\: v \right) = -4\pi^2 \left( u^2 + v^2 \right) F\left( u,\: v \right).
\end{equation*}
Let us then define the filter \(H\left( u,\: v \right)\) as
\begin{equation*}
    H\left( u,\: v \right) = -4 \pi^2 \left( u^2 + v^2 \right),
\end{equation*}
or, with respect to the center of the frequency rectangle,
\begin{equation*}
    H\left( u,\: v \right) = -4 \pi^2 \left( \left( u - P/2 \right)^2 + \left( v - Q/2 \right)^2 \right) = -4\pi D^2\left( u,\: v \right).
\end{equation*}
Then, the Laplacian of an image \(f\left( x,\: y \right)\) is given by
\begin{equation*}
    \nabla^2 f\left( x,\: y \right) = \mathscr{F}^{-1}\left\{ H\left( u,\: v \right) F\left( u,\: v \right) \right\}.
\end{equation*}
Then for the enhanced image
\begin{equation*}
    g\left( x,\: y \right) = f\left( x,\: y \right) + c \nabla^2 f\left( x,\: y \right),
\end{equation*}
\(c = -1\), so that in the frequency domain, this is equivalent to
\begin{equation*}
    G\left( u,\: v \right) = F\left( u,\: v \right) - H\left( u,\: v \right) F\left( u,\: v \right).
\end{equation*}
\subsection{Selective Filtering}
Bandreject and bandpass filters can be used to selectively filter
specific frequency bands in an image. A bandreject filter is defined as
\begin{equation*}
    H\left( u,\: v \right) =
    \begin{cases}
        0 & \text{if } D_0 - \frac{W}{2} \leq D\left( u,\: v \right) \leq D_0 + \frac{W}{2}, \\
        1 & \text{otherwise},
    \end{cases}
\end{equation*}
where \(D_0\) is the centre frequency and \(W\) is the width of the
band. A bandpass filter is defined as the complement of the bandreject
filter:
\begin{equation*}
    H_{\text{BP}}\left( u,\: v \right) = 1 - H_{\text{BR}}\left( u,\: v \right).
\end{equation*}
\subsubsection{Butterworth Bandreject Filters}
A Butterworth bandreject filter is defined as
\begin{equation*}
    H\left( u,\: v \right) = \frac{1}{1 + \left( \frac{D\left( u,\: v \right)W}{D^2\left( u,\: v \right) - D_0^2} \right)^{2n}},
\end{equation*}
where \(n\) is the order of the filter.
\subsubsection{Gaussian Bandreject Filters}
A Gaussian bandreject filter is defined as
\begin{equation*}
    H\left( u,\: v \right) = 1 - \exp{\left( - \left( \frac{D^2\left( u,\: v \right) - D_0^2}{D\left( u,\: v \right)W} \right)^2 \right)},
\end{equation*}
where \(D_0\) is the centre frequency and \(W\) is the width of the band.
\subsubsection{Notch Filters}
Notch filters are used to remove frequencies in a predefined neighbourhood
of the frequency rectangle (rather than being centred at the origin).
Zero-phase-shift filters must be symmetric about the origin, so a notch
filter transfer function with center frequencies \(\left( u_0,\: v_0 \right)\)
must have a corresponding notch filter transfer function with center
frequencies \(\left( -u_0,\: -v_0 \right)\). The transfer function of a
notch filter is constructed by multiplying the transfer functions of two
highpass filter transfer functions whose centers have been translated
to the centers of the notches. The general form for a notch filter with
\(Q\) notches is
\begin{equation*}
    H\left( u,\: v \right) = \prod_{k=1}^Q H_k\left( u,\: v \right) H_{-k}\left( u,\: v \right),
\end{equation*}
where \(H_k\left( u,\: v \right)\) and \(H_{-k}\left( u,\: v \right)\)
are the transfer functions of the highpass filters centred at the
\(k\)th notch and its negative, respectively.
\section{Colour}
Colour is a perceptual property of light that is produced when white
light is passed through a prism. The wavelengths of light that are
produced comprise the \textbf{electromagnetic spectrum}, in which the
human eye can discern wavelengths between \qtyrange[range-phrase={ and }]{400}{700}{nm}.
\subsection{Colour Fundamentals}
The human eye contains three types of cones that can be roughly
classified as being sensitive to three principal sensing regions:
\begin{itemize}
    \item \textbf{Short-wavelength cones (S-cones)}: 2\% of cones that are sensitive to blue light (\qty{435.8}{nm}). These are also the most sensitive cones, despite only making up 2\% of the total.
    \item \textbf{Medium-wavelength cones (M-cones)}: 33\% of cones that are sensitive to green light (\qty{546.1}{nm}).
    \item \textbf{Long-wavelength cones (L-cones)}: 64\% of cones that are sensitive to red light (\qty{700}{nm}).
\end{itemize}
These \textbf{primary} colours can be combined to produce the
\textbf{secondary} colours of light:
\begin{itemize}
    \item \textbf{Cyan}: A combination of blue and green light.
    \item \textbf{Magenta}: A combination of blue and red light.
    \item \textbf{Yellow}: A combination of green and red light.
\end{itemize}
Mixing the primary colours of light produces white light. The primary
colours of \textbf{pigment} are the secondary colours of light, and
mixing these produces black pigment. The primary colours are also
referred to as the \textbf{additive primaries}, while the secondary
colours are referred to as the \textbf{subtractive primaries}.
\subsubsection{Characterising Colour}
Colour can be characterised by three attributes:
\begin{itemize}
    \item \textbf{Brightness}: The intensity of the colour.
    \item \textbf{Hue}: The dominant wavelength of light.
    \item \textbf{Saturation}: The relative purity of the colour, or the amount of white light mixed with a hue.
    Hence, the degree of saturation is inversely proportional to the amount of white light mixed with a hue.
\end{itemize}
Hue and saturation can also be combined to form the \textbf{chromaticity}
of a colour.
\subsection{Colour Models}
Colour models, or color spaces, are coordinate system for representing
colours mathematically. Some common colour spaces include:
\begin{itemize}
    \item \textbf{RGB}: The most common colour space, which represents
          colours as a combination of red, green, and blue intensities.
          RGB is used in displays and cameras.
    \item \textbf{CMY(K)}: A subtractive colour model that represents
            colours as a combination of cyan, magenta, and yellow
            intensities, with an optional black intensity.
            CMYK is used in colour printing.
    \item \textbf{HSI}: A cylindrical colour space that represents
    the hue, saturation, and intensity of a colour. This is also known
    as HSV (value) or HSL (lightness), where the intensity may be scaled
    to a different range.
\end{itemize}
\subsubsection{RGB Colour Model}
The RGB colour model can be represented as a unit cube, with a corner at
the origin, with the primary colours of red, green, and blue at the
vertices:
\begin{itemize}
    \item \(\left( 1,\: 0,\: 0 \right)\): Red.
    \item \(\left( 0,\: 1,\: 0 \right)\): Green.
    \item \(\left( 0,\: 0,\: 1 \right)\): Blue.
\end{itemize}
The colour black is at the origin, while the colour white is at the
opposite corner. Grayscale colours lie along the diagonal of the cube
which intersects these two points. Note that in this model, colour
values are normalised to the range \(\interval{0}{1}\).
\subsubsection{CMY(K) Colour Model}
The CMY(K) colour model exists on the same cube as the RGB colour model,
but with the primary colours of cyan, magenta, and yellow at the
vertices:
\begin{itemize}
    \item \(\left( 0,\: 1,\: 1 \right)\): Cyan.
    \item \(\left( 1,\: 0,\: 1 \right)\): Magenta.
    \item \(\left( 1,\: 1,\: 0 \right)\): Yellow.
\end{itemize}
Mathematically, CMYK colours are the complement of RGB colours:
\begin{equation*}
    \begin{bmatrix}
        C \\
        M \\
        Y
    \end{bmatrix} = \begin{bmatrix}
        1 \\
        1 \\
        1
    \end{bmatrix} - \begin{bmatrix}
        R \\
        G \\
        B
    \end{bmatrix}.
\end{equation*}
\subsubsection{HSI Colour Model}
The HSI colour model is a cylindrical colour space that represents
colours as a combination of hue, saturation, and intensity. The hue
value is the angle around this cylinder, with red at \(0^{\circ}\), green at
\(120^{\circ}\), and blue at \(240^{\circ}\). The intensity value is the
distance along the intensity axis in the RGB cube, while the saturation
value is the distance from the intensity axis.
\begin{align}
    \theta & = \arccos\left( \frac{2R - G - B}{2\sqrt{\left( R - G \right)^2 + \left( R - B \right)\left( G - B \right)}} \right), \\
    H & = \begin{cases}
        \frac{\theta}{360^{\circ}} & \text{if } B \leqslant G, \\
        1 - \frac{\theta}{360^{\circ}} & \text{if } B > G,
    \end{cases} \\
    S & = 1 - \frac{3}{R + G + B} \min\left( R,\: G,\: B \right), \\
    I & = \frac{R + G + B}{3}.
\end{align}
To convert from HSI to RGB, we can use the following equations. In the
RG sector (\(0^{\circ} \leqslant H \leqslant 120^{\circ}\)):
\begin{align*}
    B & = I\left( 1 - S \right), \\
    R & = I\left( 1 + \frac{S\cos H}{\cos\left( 60^{\circ} - H \right)} \right), \\
    G & = 3I - \left( R + B \right).
\end{align*}
In the GB sector (\(120^{\circ} \leqslant H \leqslant 240^{\circ}\)):
\begin{align*}
    R & = I\left( 1 - S \right), \\
    G & = I\left( 1 + \frac{S\cos\left( H - 120^{\circ} \right)}{\cos\left( 60^{\circ} - \left( H - 120^{\circ} \right) \right)} \right), \\
    B & = 3I - \left( R + G \right).
\end{align*}
In the BR sector (\(240^{\circ} \leqslant H \leqslant 360^{\circ}\)):
\begin{align*}
    G & = I\left( 1 - S \right), \\
    B & = I\left( 1 + \frac{S\cos\left( H - 240^{\circ} \right)}{\cos\left( 60^{\circ} - \left( H - 240^{\circ} \right) \right)} \right), \\
    R & = 3I - \left( G + B \right).
\end{align*}
\subsection{Colour Image Acquisition}
Colour images are acquired using monochromatic charge-coupled devices
(CCDs) that are sensitive to red, green, and blue light. These devices
are arranged in a Bayer filter mosaic, which consists of a \(2 \times 2\)
grid of sensors with one red, one blue, and two green sensors.
\begin{equation*}
    \begin{bmatrix}
        G & R \\
        B & G
    \end{bmatrix}
\end{equation*}
Here two green sensors are used to improve the resolution of the green
channel, which is the most sensitive to the human eye. When demosaicking
the image, the colour channels are extracted from the mosaic, and
missing entries are interpolated using the values of neighbouring
pixels in that channel.
\begin{align*}
    \begin{bmatrix}
        G_{11} & R_{12} \\
        B_{21} & G_{22}
    \end{bmatrix}
    \to
    \begin{bmatrix}
        - & R_{12} \\
        - & -
    \end{bmatrix}
    +
    \begin{bmatrix}
        G_{11} & - \\
        - & G_{22}
    \end{bmatrix}
    +
    \begin{bmatrix}
        - & - \\
        B_{21} & -
    \end{bmatrix}
\end{align*}
\subsubsection{Pixel Depth}
The pixel depth of an image is the number of bits used to represent the
colour of each pixel. The pixel depth determines the total number of
colours that can be represented in an image. For an image with \(n\)
bits per channel, the total number of colours is given by \(2^{3n}\).
For example, an image with 8 bits per channel has a pixel depth of 24
bits and can represent \(2^{24} = \num{16777216}\) unique colours.
\subsection{Colour Image Processing}
Colour images can be processed using the same techniques as grayscale
images, but with the additional consideration of the colour channels.
Transformations can either be applied to each channel independently or
to individual channels, such as the hue, saturation, and intensity
channels in the HSI colour space. We can define colour image
transformations in the same way as for grayscale images:
\begin{equation*}
    g\left( x,\: y \right) = T\left( f\left( x,\: y \right) \right),
\end{equation*}
where \(f\left( x,\: y \right)\) is the input image, \(g\left( x,\: y \right)\)
is the output image, and \(T\) is the transformation function.
\subsubsection{Colour Intensity Transformations}
Intensity transforms take the form:
\begin{equation*}
    g\left( x,\: y \right) = k f\left( x,\: y \right),
\end{equation*}
\begin{itemize}
    \item RGB: \(s_i = k r_i\) for \(i = 1, 2, 3\).
    \item HSI: \(s_1 = r_1\), \(s_2 = r_2\), and \(s_3 = k r_3\).
    \item CMY: \(s_i = k r_i + \left( 1 - k \right)\) for \(i = 1, 2, 3\).
\end{itemize}
\subsubsection{Colour Complements}
Colour complements are hues directly opposite each other on the colour
circle. Complements are analogous to negative images in grayscale
images and can be useful for enhancing detail in dark regions.
\subsubsection{Colour Slicing}
Colour slicing is a technique used to highlight specific colours in an
image to separate objects from surroundings.
\subsubsection{Colour Tonal Transformations}
Colour tonal transformations are used to adjust the tonal range of an
image. These transformations can be used to boost contrast in an image
using power-law transformations.
\subsubsection{Colour Balancing (White Balancing)}
An image is colour imbalanced when a known white point in the image does
not contain equal RGB or CMY components. Colour balancing is used to
correct these imbalances by adjusting the channels to make such points
truly white.
\subsubsection{Colour Histogram Equalisation}
Histogram equalisation can also be applied to the intensity channel of
HSI images to improve the contrast of the image without altering the
hue or saturation.
\subsubsection{Colour Image Smoothing and Sharpening}
Colour images can be smoothed or sharpened using the same techniques as
grayscale images, by applying a filter to each channel independently.
For example, an image can be smoothed by considering it's \(K\) average
neighbours:
\begin{equation*}
    \symbf{y}_{ij} = \frac{1}{K} \sum_{\left( s,\: t \right) \in S} \symbf{x}_{st},
\end{equation*}
where \(S\) is the set of \(K\) neighbours of pixel \(\left( i,\: j \right)\).
Similarly, an image can be sharpened by applying a Laplacian filter to
each channel independently. The Laplacian of a colour image \(\symbf{x}\)
is simply
\begin{equation*}
    \nabla^2 \symbf{x} = \begin{bmatrix}
        \nabla^2 x_1 \\
        \nabla^2 x_2 \\
        \nabla^2 x_3
    \end{bmatrix}
\end{equation*}
\subsubsection{Colour Image Segmentation}
Colour image segmentation is the process of partitioning an image into
regions based on colour. For HSI images, segmentation is performed on the
hue channel, and the saturation channel is used to further isolate
regions in the hue image. For RGB images, we can use distance metrics to
measure if nearby pixels are similar to a desired colour \(\symbf{a}\).
\end{document}
